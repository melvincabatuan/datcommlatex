\documentclass{beamer}

\usepackage{beamerthemeblackboard}
\usepackage{graphics}



\usepackage{beamerthemeblackboard}
\usepackage{graphics}
\usepackage{ifsym} %  
\usepackage{epsdice} % dice
\usepackage{clock} % 
\usepackage{microtype}
\usepackage{skull} %create skull
\usepackage{cancel} %create diagonal bar (cancel)
\usepackage{stackrel} %create arc?
\usepackage{bbding} %create hands and cross
\usepackage{pifont} %create circled numbers
\usepackage{lipsum} % Just some sample text
\usepackage{amsmath,amssymb,amsthm} %math functions like align
\usepackage{fancybox} %For beautiful boxes
\usepackage{xspace} % Prints a trailing space in a smart way.
\usepackage{units}
\usepackage{geometry} % change paper size
\usepackage{multicol} % Small sections of multiple columns 
\usepackage{color}


% define your own colours:
\definecolor{Red}{rgb}{1,0,0}
\definecolor{Blue}{rgb}{0,0,1}
\definecolor{Green}{rgb}{0,1,0}
\definecolor{magenta}{rgb}{1,0,.6}
\definecolor{lightblue}{rgb}{0,.5,1}
\definecolor{lightpurple}{rgb}{.6,.4,1}
\definecolor{gold}{rgb}{.6,.5,0}
\definecolor{orange}{rgb}{1,0.4,0}
\definecolor{hotpink}{rgb}{1,0,0.5}
\definecolor{newcolor2}{rgb}{.5,.3,.5}
\definecolor{newcolor}{rgb}{0,.3,1}
\definecolor{newcolor3}{rgb}{1,0,.35}
\definecolor{darkgreen1}{rgb}{0, .35, 0}
\definecolor{darkgreen2}{rgb}{0, .38, 0}
\definecolor{darkgreen}{rgb}{0, .6, 0}
\definecolor{darkred}{rgb}{.75,0,0}
\xdefinecolor{olive}{cmyk}{0.64,0,0.95,0.4}
\xdefinecolor{purpleish}{cmyk}{0.75,0.75,0,0}
 



% MY COMMANDS
\newcommand{\s}{\vspace{0.2cm}} % adds space
\newcommand{\ns}{\vspace{-0.5cm}}  % subtracts space
\newcommand{\fig}[2]{
\begin{center}
\begin{figure}
\includegraphics[scale=#1]{figures/#2}
\end{figure}
\end{center}
}

%question block with choices in column
\newcommand{\qc}[6]{
\begin{block}{\Large #1 }
\begin{enumerate}[]
\item #2
\item #3
\item #4
\item #5
\item #6
\end{enumerate} 
\end{block}
}

\newcommand{\ans}[2]{\alert<#1>{\textbf<#1>{#2}}  \only<#1>{ \textcolor {Red} \checkmark }} 
\newcommand{\dquote}[1]{\ding{125} \emph{#1} \ding{126}} %decorative quote
\newcommand{\true}{\text{\bf T}} %true 
\newcommand{\false}{\text{\bf F}} %false
%formula block
\newcommand{\f}[1]{
\begin{center}
\shadowbox{ $ #1 $}
\end{center}
}



 

\begin{document}

% set handwritten font, necessary packages are loaded in beamerthemeblackboard.sty
\ECFAugie

\begin{frame}

\begin{center}
\begin{figure}
\includegraphics[scale=0.3]{figures/dlsulogo}
\end{figure}
\end{center}
\ns

\title{DATCOMM  \\ \underline{Local Area Network} \\ \underline{Technologies}}
\author{Engr. Melvin K. Cabatuan, MsE}
\date{February 2013}
\institute{De La Salle University}
\maketitle
\end{frame}

\begin{frame}
\frametitle{Objectives}
\begin{itemize} 
\Large
\item <1-> To briefly discuss the technology of dominant wired LANs- Ethernet, and other LAN media.
\item <2-> Describe Media Access Control (MAC) and Carrier Sense Multiple Access/Collision Detection (CSMA/CD)
\item <3-> Explain Address Resolution Protocol (ARP) and Bridges.
\item <4-> Discuss Switched Ethernet and \\ Virtual LAN (VLAN).

\end{itemize}
\end{frame}

 


\begin{frame}
\frametitle{Local Area Network}
\framesubtitle{LAN}
\begin{itemize} 
\Large
\item <1-> A computer network that is designed for a limited geographic area such as a building or a campus. 
\item <2-> LAN technologies: Ethernet, token ring, token bus, FDDI, and ATM LAN.
\end{itemize}
\fig{0.65}{network_lan}
\end{frame}


\begin{frame}
\frametitle{\huge IEEE Project 802}
\framesubtitle{IEEE standard for LANs}
\begin{itemize} 
\Large
\item <1-> specify functions of the physical and data link layer of major LAN protocols.
\item <2-> subdivided the data link layer into two sublayers: logical link control (LLC) and media access control (MAC).
\end{itemize}
\fig{0.55}{network_802}
\end{frame}


\begin{frame}
\frametitle{Ethernet} 
\begin{itemize} 
\Large
\item <1-> was created in 1976 at Xerox's Palo Alto Research Center (PARC). 
\item <2-> de facto standard technology that is used for connecting LANs.
\item <3-> first implemented by a group called DIX (Digital, Intel, and Xerox). 
\end{itemize}
\fig{0.6}{network_etherevolution}
\end{frame}


\begin{frame}
\frametitle{\huge Ethernet (802.3) Frame} 
\begin{itemize} 
\Large
\item <1-> Preamble \\ contains 7 bytes (56 bits) of alternating $0$s and $1$s that alerts the receiving system to the coming frame and enables it to synchronize its input timing.
\end{itemize}
\fig{0.55}{network_etherframe}
\end{frame} 


\begin{frame}
\frametitle{\huge Ethernet (802.3) Frame} 
\begin{itemize} 
\Large
\item <1-> Start frame delimiter (SFD) \\ (1 byte: 10101011) signals the beginning of the frame; the last 2 bits are $11$ and alert the receiver that the next field is the destination address.
\end{itemize}
\fig{0.55}{network_etherframe}
\end{frame} 
 

\begin{frame}
\frametitle{\huge Ethernet (802.3) Frame} 
\begin{itemize} 
\Large
\item <1-> Destination address (DA) \\ is 6 bytes and contains the physical address of the destination station/s to receive the packet.
\item <2-> Source address (SA) \\ is 6 bytes and contains the physical address of the sender of the packet.
\end{itemize}
\fig{0.55}{network_etherframe}
\end{frame} 


\begin{frame}
\frametitle{\huge Ethernet (802.3) Frame} 
\begin{itemize} 
\Large
\item <1-> Length or Type \\  
$\circ$ 802.3: length field to define the number of bytes in the data field or \\
$\circ$ Ethernet: type field to define the upper-layer protocol using the MAC frame.
\end{itemize}
\fig{0.55}{network_etherframe}
\end{frame} 


\begin{frame}
\frametitle{\huge Ethernet (802.3) Frame} 
\begin{itemize} 
\Large
\item <1-> Data \\ $\circ$ carries data encapsulated from the upper-layer protocols; \\ 
$\circ$ a minimum of 46 and a maximum of 1500 bytes.
\end{itemize}
\uncover<2->{
\fig{0.55}{network_etherframe2}
}	
\end{frame} 



 \frame{\frametitle{\HandRight \: Understand}
\Large
What if the upper-layer packet is less than the minimum 46 bytes?
\begin{itemize}
\item <2-> Padding is added to make up the difference.
\end{itemize}
 }


\begin{frame}
\frametitle{\huge Ethernet (802.3) Frame} 
\begin{itemize} 
\Large
\item <1-> Cyclic Redundancy Check (CRC) \\ verifies that the data that left the source computer did not change at all during the transmission.
\end{itemize}
\fig{0.55}{network_etherframe2}
\end{frame} 



 \frame{\frametitle{\HandRight \: Understand}
\Large
The 802.3 standard defines the maximum length of a frame (without preamble and SFD field) as $1518$ bytes. Give the historical
reasons for this restriction.
\begin{itemize}
\item <2-> Memory was very expensive when Ethernet was designed: a maximum length restriction helped to reduce the size of the buffer.
\item <3-> It prevents one station from monopolizing the shared medium, blocking other stations that have \\ data to send.
\end{itemize}
 }


\begin{frame}
\frametitle{\huge MAC Address} 
\framesubtitle{also referred to as the data link address or physical address}
\begin{itemize} 
\Large
\item <1-> a 6 bytes (48 bits) physical address applied to the network interface card (NIC) by the manufacturer during production.
\end{itemize}
\fig{0.35}{network_adapter}
\end{frame} 



\begin{frame}
\frametitle{\huge MAC Address} 
\begin{itemize} 
\Large
\item <1-> normally written in hexadecimal notation, with a colon between the bytes.
\end{itemize}
\fig{0.8}{network_macaddress}
\begin{itemize} 
\Large
\item <2-> Ex. Ethernet MAC address \\
\qquad   $4$A$:30:10:21:10:1$A 
\fig{0.5}{network_macaddress2}
\end{itemize}
\end{frame} 


\begin{frame}
\frametitle{\huge Source and Destination Addressing Modes} 
\begin{itemize} 
\Large
\item <1-> Source address is always a unicast address - the frame comes from only one station.
\item <2-> Destination address can be unicast, multicast, or broadcast.
\end{itemize}
\fig{0.6}{network_unimulti}
\end{frame} 



 \frame{\frametitle{\HandRight \: Exercise}
\Large
Define the type of the following destination addresses:
\begin{itemize}
\item <2-> $4A:30:10:21:10:1A$
\item <3-> $47:20:1B:2E:08:EE$
\item <4-> $FF:FF:FF:FF:FF:FF$ 
\end{itemize}
\uncover<5->{
Solution: Refer to the second hexadecimal digit from the left: \\ 
\qquad even ==> unicast; \\ 
\qquad odd ==> multicast; \\ 
\qquad all F's ==> Broadcast
}
}

 
\begin{frame}
\frametitle{\huge Transmission of Addresses} 
\begin{itemize} 
\Large
\item <1-> transmission is left-to-right, byte by byte; however, for each byte, the least significant bit is sent first and the most significant bit is sent last.
\item <2-> Ex. Show how the address $47:20:1B:2E:08:EE$ is sent out on line.
\end{itemize}
\uncover<3->{
\fig{0.6}{network_addtrans}
}
\end{frame} 



\begin{frame}
\frametitle{CSMA/CD} 
\framesubtitle{Carrier Sense Multiple Access with Collision Detection}
\begin{itemize} 
\Large
\item <1-> access method for traditional Ethernet (10-Mbps) that  senses the medium before trying to use it.
\item <2-> Ethernet stations can be connected together using a physical bus or star topology but its logical topology is always a bus.
\end{itemize}
\end{frame} 


\begin{frame}
\frametitle{\huge CSMA Collision} 
\fig{0.6}{network_csma}
\end{frame} 


\begin{frame}
\frametitle{CSMA/CD} 
\framesubtitle{Carrier Sense Multiple Access with Collision Detection Algorithm}
\fig{0.55}{network_csmacd}
\end{frame} 


\begin{frame}
\frametitle{CSMA/CD} 
\framesubtitle{Minimum Frame Size}
 \begin{itemize} 
\Large
\item <1-> a restriction on the frame size is required.
\item <2-> before sending the last bit of the frame, the sending station must detect a collision and abort.
\item <3-> thus, transmission time $T_{fr}$ must be at least two times the maximum propagation time $T_p$.
\end{itemize}
\end{frame}


 \frame{\frametitle{\HandRight \: Exercise}
\Large
In the standard Ethernet, if the maximum propagation time is $25.6 \mu s$, what is the minimum size of the frame?
\begin{itemize}
\item[] <2->  $T_{fr} = 2 \times T_p = 51.2 \mu s$
\item[] <3-> $10$ Mbps $\times 51.2 \mu s = 512$ bits or $64$ bytes
\item <4-> This is the minimum size of the frame for Standard Ethernet.
\end{itemize}
 }


\begin{frame}
\frametitle{CSMA/CD} 
\framesubtitle{Flow Diagram}
\fig{0.45}{network_csmacd2}
\end{frame}


\begin{frame}
\frametitle{\huge Standard Ethernet}   
\framesubtitle{Implementation}
\fig{0.45}{network_ethernet2}
\end{frame}


\begin{frame}
\frametitle{\HandPencilLeft  \: Reading Assignment}
\begin{itemize}
\huge
\item  <1-> Report about the IEEE 802.3 Standard, in your own words. 
\item  <1-> Submit through www.turnitin.com.
 \end{itemize}
\end{frame}

\begin{frame}
\frametitle{\huge Cable Specifications}   
\fig{0.5}{network_ethernetcable}
\end{frame}



\begin{frame}
\frametitle{\huge Coaxial Cable}   
\fig{0.55}{network_coax}
\end{frame}


\begin{frame}
\frametitle{\huge Shielded Twisted Pair (STP)}   
\fig{0.5}{network_twisted}
\end{frame}


\begin{frame}
\frametitle{\huge Unshielded Twisted Pair (UTP)}   
\fig{0.5}{network_utp2}
\end{frame}


 \frame{\frametitle{\HandRight \: Readings}
\Large
It is the standards body that creates the Physical layer specifications for Ethernet.
\begin{itemize} 
\item <2->  EIA/TIA (Electronic Industries Association and the newer Telecommunications Industry Association) 
\item <3-> EIA/TIA specifies that Ethernet use a registered jack (RJ) connector with a 4 5 wiring sequence on unshielded twisted-pair (UTP) cabling (RJ-45). 
\end{itemize}
}




\begin{frame}
\frametitle{\huge UTP Connections (RJ-45)} 
 \begin{itemize} 
\Large
\item <1-> RJ-45 connector is clear so you can see the eight colored wires that connect to the connector's pins. These wires are twisted into four pairs. 
\item <2-> Four wires (two pairs) carry the voltage and are considered \underline{tip}. The other four wires are grounded and are called \underline{ring}.   
\end{itemize}
\end{frame}


\begin{frame}
\frametitle{\huge UTP Connections (RJ-45)}   
\framesubtitle{8-pin modular connector}
\fig{0.5}{network_utp}
\end{frame}


\begin{frame}
\frametitle{\huge Ethernet Cabling} 
\Large
Straight-through cable: used to connect
 \begin{itemize} 
\item <1-> Host to switch or hub. 
\item <2-> Router to switch or hub
\end{itemize}
\fig{0.6}{network_straight}
\end{frame}

\begin{frame}
\frametitle{\huge Ethernet Cabling} 
\Large
Straight-through cable: wires on both cable ends are in the same order.  
 
\fig{0.6}{network_straight2}
\end{frame}

\begin{frame}
\frametitle{\huge Ethernet Cabling} 
\Large
Crossover Cable: used to connect
 \begin{itemize} 
\item <1-> Switch to switch 
\item <2-> Hub to hub
\item <3-> Host to host
\item <4-> Hub to switch 
\item <5-> Router direct to host
\end{itemize}
\fig{0.6}{network_cross}
\end{frame}

\begin{frame}
\frametitle{\huge Ethernet Cabling} 
\Large
Crossover Cable: wires on each end of the cable are crossed - Transmit to Receive and Receive to Transmit on each side, for both tip and ring.  
\fig{0.55}{network_cross2}
\end{frame}



\begin{frame}
\frametitle{\huge Ethernet Cabling} 
\framesubtitle{Summary}
\fig{0.4}{network_cabling}
\end{frame}



\begin{frame}
\frametitle{\huge Fast Ethernet (802.3u)} 
\Large
 \begin{itemize} 
\item <1-> designed to compete with LAN protocols such as FDDI or Fiber Channel. 
\item <2-> upgrade the data rate to 100 Mbps.
\item <3-> backward-compatible with Standard Ethernet.
\item <4-> same frame format and 48-bit address.
\item <5-> same minimum and maximum frame lengths.
\end{itemize}
\end{frame}


\begin{frame}
\frametitle{\huge Fast Ethernet (802.3u) Implementation} 
\fig{0.5}{network_fastethernet}
\end{frame}


\begin{frame}
\frametitle{\huge Fast Ethernet (802.3u)} 
\Large
 \begin{itemize} 
\item <1-> MAC sublayer was kept untouched
\item <2-> star topology: half duplex and full duplex
\item <3-> access method is the same (CSMA/CD) for the half-duplex
\item <4-> \underline{autonegotiation} allows two devices to negotiate the mode or data rate of operation.
\end{itemize}
\end{frame}


\begin{frame}
\frametitle{\huge Address Resolution Protocol (ARP)} 
\Large
 \begin{itemize} 
\item <1-> accepts a logical address from the IP protocol, then, identify and place the source and destination MAC address in the frame
\item <2-> operates at the Internet layer, but the the MAC address is attached at the Network Access layer.
\item <3-> maps a logical address to its corresponding physical address
\end{itemize}
\end{frame}


\begin{frame}
\frametitle{\huge Address Resolution Protocol (ARP)} 
\framesubtitle{Position of ARP in TCP/IP protocol suite}
\fig{0.55}{network_arp}
 \begin{itemize} 
\Large
\item <2-> Why do we need ARP?
\end{itemize}
\end{frame}


\begin{frame}
\frametitle{\huge ARP Operation} 
\framesubtitle{ARP request is broadcast}
\dquote{If this is your IP address, send me your \\ MAC address.}
\fig{0.45}{network_arpop1}
\end{frame}


\begin{frame}
\frametitle{\huge ARP Operation} 
\framesubtitle{ARP reply is unicast}
\dquote{This is my MAC address.}
\fig{0.45}{network_arpop2}
\end{frame}




\begin{frame}
\frametitle{\huge ARP Packet Format} 
\fig{0.55}{network_arppacket}
\end{frame}


\begin{frame}
\frametitle{\huge ARP Packet Format} 
\Large
 \begin{itemize} 
\item <1->  Hardware type \\ $16$-bit field defining the type of the network on which ARP is running. Each LAN has been assigned. \\ 
\quad Ex. Ethernet: type $1$
\item <2-> Protocol type \\ $16$-bit field defining the protocol. \\ 
\quad Ex. IPv4 protocol: $0800_{16}$  
\item <3-> Hardware length \\  8-bit field defining the length of the physical address in bytes. \\ 
\quad Ex. Ethernet: $6$ bytes 
\end{itemize}
\end{frame}


\begin{frame}
\frametitle{\huge ARP Packet Format} 
\Large
 \begin{itemize} 
\item <1->  Protocol Length \\ $8$-bit field defining the length of the logical address in bytes. \\ 
\quad Ex. IPv4 protocol: $4$
\item <2-> Operation \\ 16-bit field defining the type of packet\\
\quad Ex. ARP request ($1$), ARP reply ($2$).
\item <3-> Sender hardware address \\  variable-length field defining the physical address of the sender. \\ 
\quad Ex. Ethernet: $6$ bytes 
\end{itemize}
\end{frame}


\begin{frame}
\frametitle{\huge ARP Packet Format} 
\Large
 \begin{itemize} 
\item <1->  Sender protocol address \\ variable-length field defining the logical address of the sender. \\ 
\quad Ex. IPv4 protocol: $4$ bytes
\item <2-> Target hardware address \\ variable-length field defining the physical address of the target.\\
\quad Ex. Ethernet: $6$ bytes 
\item <3-> Target protocol address \\ variable-length field defining the logical address of the target. \\ 
\quad Ex. IPv4 protocol: $4 bytes$
\end{itemize}
\end{frame}


\begin{frame}
\frametitle{\huge Encapsulation of ARP packet} 
\Large
An ARP packet is encapsulated directly into a data link frame.
\fig{0.5}{network_arpenc}
\end{frame}


\begin{frame}
\frametitle{\huge ARP Usage Scenarios} 
\fig{0.45}{network_arpscenario1}
\end{frame}


\begin{frame}
\frametitle{\huge ARP Usage Scenarios} 
\fig{0.45}{network_arpscenario2}
\end{frame}

 \frame{\frametitle{\HandRight \: Exercise}
\Large
A host with IP address $130.23.43.20$ and physical address $B2:34:55:10:22:10$ has a packet to send to another host with IP address $130.23.43.25$ and physical address $A4:6E:F4:59:83:AB$ (which is unknown to the first host). The two hosts are on the same Ethernet network. Show the ARP request and reply packets encapsulated in Ethernet frames.
}


\begin{frame}
\frametitle{\huge ARP Exercise Request} 
\fig{0.4}{network_arprequest}
\end{frame} 


\begin{frame}
\frametitle{\huge ARP Exercise Reply} 
\fig{0.4}{network_arpreply}
\end{frame} 


\begin{frame}
\frametitle{\huge Proxy ARP Router} 
\Large
 \begin{itemize} 
\item <1-> acts on behalf of a set of hosts.
\item <2-> whenever a router running a proxy ARP receives an ARP request looking for the IP address of one of its hosts, the router sends an ARP reply announcing its \underline{own} hardware (physical) address.
\end{itemize}
\end{frame}


   \begin{frame}
\frametitle{\huge Proxy ARP Router} 
\fig{0.45}{network_arpproxy}
After the router receives the actual IP packet, it  \\ sends the packet to the appropriate host or router.
\end{frame}


   \begin{frame}
\frametitle{\huge Connecting Devices} 
\Large
 \begin{itemize} 
\item <1-> To connect LANs and WANs together we use connecting devices.
\item <2-> Ex. Repeaters (or hubs), Bridges (or two-layer switches), and Routers (or three-layer switches).
\end{itemize}
 \fig{0.55}{network_connections}
\end{frame}


   \begin{frame}
\frametitle{\huge Repeater} 
\Large
\fig{0.65}{network_lanhub}
A repeater forwards every bit; it has no filtering capability.
\end{frame}

\begin{frame}
\frametitle{\huge Bridge} 
\framesubtitle{Two-layer switch}
\Large
\begin{itemize} 
\item <1-> operates in both the physical and the data link layers.
\item <2-> PHY: regenerates the signal it receives. 
\item <3-> DLL: check the MAC addresses (source and destination) contained in the frame.
\item <4-> has a table used in filtering decisions.
\end{itemize}
\end{frame}






\begin{frame}
\frametitle{\huge Bridge} 
\framesubtitle{Example}
\Large
The bridge consults its table to find the departing port.
\fig{0.55}{network_bridge}
\end{frame}



\begin{frame}
\frametitle{\huge Transparent Bridge} 
\Large
\begin{itemize} 
\item <1-> a bridge in which the stations are completely unaware of the bridge's existence.
\item <2-> reconfiguration of the stations is unnecessary when added or deleted.
\item <3-> forwarding function with dynamic forwarding table
\end{itemize}
\end{frame}


\begin{frame}
\frametitle{\huge Bridge Learning} 
\fig{0.6}{network_bridge2}
\end{frame}


\begin{frame}
\frametitle{\huge Switched LAN} 
\framesubtitle{Traditional}
\fig{0.5}{network_swlan}
\end{frame}


\begin{frame}
\frametitle{\huge Switched LAN} 
\framesubtitle{Contemporary}
\fig{0.55}{network_swlan2}
\end{frame}


\begin{frame}
\frametitle{\huge Bridging vs. LAN Switching} 
\Large
\begin{itemize} 
\item <1-> Bridges are software based, while switches are hardware based (ASIC for filtering)
\item <2-> A switch can be viewed as a multiport bridge.
\item <3-> Switches have a higher number of ports than most bridges.
\item <4-> Both bridges and switches forward layer 2 broadcasts.
\end{itemize}
\end{frame}


\begin{frame}
\frametitle{\huge Layer 2 Switch Functions} 
\Large
\begin{itemize} 
\item <1-> Address learning \\
remember the source hardware address of each frame received, and save in forward/filter table.
\item <2-> Forward/filter decisions \\
When a frame is received, the switch looks at the destination hardware address and finds the exit interface.
\item <3-> Loop avoidance \\
stop network loops while still permitting redundancy.
\end{itemize}
\end{frame}
  

\begin{frame}
\frametitle{\huge Address Learning} 
\fig{0.55}{network_swlearn}
\end{frame}


\begin{frame}
\frametitle{\huge Forward/Filter Decisions}
Host A sends a data frame to Host D. What will the switch do when it receives the frame from Host A? 
\fig{0.55}{network_filter}
\end{frame}

\begin{frame}
\frametitle{\huge Network Redundancy} 
\framesubtitle{Importance and Problem}
\fig{0.45}{network_redundancy}
\end{frame}


\begin{frame}
\frametitle{\huge Loop Avoidance} 
\framesubtitle{Spanning Tree Protocol (STP)}
\Large
" All root ports forward, \\ All nonroot ports block ".
\fig{0.6}{network_stp}
\end{frame}


\begin{frame}
\frametitle{\huge Virtual Private Network}
\framesubtitle{VPN}
\begin{itemize} 
\Large
\item <1-> It enables a host computer to send and receive data across shared or public networks as if it were a private network with all the functionality
\end{itemize}
\fig{0.3}{network_vpn}
\end{frame}


\begin{frame}
\frametitle{\huge Network Security Issue}
\begin{itemize} 
\Large
\item <1-> Ensure confidentiality through use of \\
\quad $\circ$ User authentication \\
\quad $\circ$ Data encryption
\end{itemize}
\fig{0.55}{network_security}
\end{frame}



\begin{frame}
\frametitle{\huge Virtual Private Networks}
\Large
\begin{itemize} 
\item <1-> network connection that uses the Internet to give users or branch offices secure access to a company's network resources.
\item <2-> use encryption technology to ensure that communication is private and secure
\item <3-> Privacy is achieved by creating a "tunnel" between the VPN client and VPN server.
\end{itemize}
\end{frame}


\begin{frame}
\frametitle{\huge Virtual Private Network}
A \underline{tunnel} is created by encapsulation, in which the inner packet containing the data is encrypted and the outer headers contain the unencapsulated addresses.
\fig{0.45}{network_tunnel}
\end{frame}


\begin{frame}
\frametitle{\huge VPN Types/ Benefits}
\Large
\begin{itemize} 
\item <1-> Remote access VPNs \\ Enable mobile users to connect with corporate networks securely wherever an Internet connection is available.
\item <2-> Site-to-site VPNs or intranet \\ Allow multiple sites to maintain permanent secure connections via the Internet instead of using expensive WAN links.
\end{itemize}
\end{frame}



\begin{frame}
\frametitle{\huge VPN Types/ Benefits}
\Large
\begin{itemize} 
\item <1-> Reduce costs by using the ISP's support services instead of paying for more expensive WAN support.
\item <2-> Eliminate the need to support dial-up remote access, which is a higher-cost solution requiring more personnel.
\end{itemize}
\end{frame}


\begin{frame}
\frametitle{\huge IP Security}
\framesubtitle{IPSec}
\Large
\begin{itemize} 
\item <1-> a collection of protocols designed by the Internet Engineering Task Force (IETF) to provide security for a packet at the network level..
\item <2-> helps create authenticated and confidential packets for the IP layer.
\item <3-> operates in one of two different modes: transport or tunnel mode.
\end{itemize}
\end{frame}


\begin{frame}
\frametitle{\huge IPSec Transport Mode}
\Large
IPSec in transport mode does not protect the IP header;\\ it only protects the information coming from the transport layer.
\fig{0.5}{network_ipsectrans}
It is used when we need host-to-host (end-to-end) protection of data.
\end{frame}


\begin{frame}
\frametitle{\huge IPSec Tunnel Mode}
\Large
IPSec in tunnel mode protects the original IP header.
\fig{0.5}{network_ipsectunnel}
It is used between two routers, between a host and a router, or between a router and a host.
\end{frame}


\begin{frame}
\frametitle{\huge Tunnel vs. Transport Mode}
\Large
$\circ$ In transport mode, the IPSec layer comes between the transport layer and the network layer. \\
$\circ$ In tunnel mode, the flow is from the network layer to the IPSec layer and then back to the network layer again.
\fig{0.5}{network_tunnelvstrans}
\end{frame}




\frame{\frametitle{Other Terms}
\begin{itemize} 
\item 1000BASE-CX, 1000BASE-LX, 1000BASE-SX, 1000BASE-T \\  
The IEEE 802.3 standards for Ethernet implementation with 1-Gbps data rate.
\item 100BASE-FX, 100BASE-T4, 100BASE-TX, 100BASE-X \\ 
The IEEE 802.3 standards for Fast Ethernet implementation with 100-Mbps data rate.
\item 10BASE2, 10BASE5, 10BASE-F, 10BASE-E, 10BASE-L \\  
The IEEE 802.3 standard for Thin Ethernet with 10-Mbps data rate.  
\end{itemize}
}

\frame{\frametitle{Other Terms}
\begin{itemize} 
\item Address Resolution Protocol (ARP) \\
In TCP/IP, a protocol for obtaining the physical address of a node when the Internet address is known.
\item  Address space  \\ 
The total number of addresses used by a protocol.
\item Bandwidth \\  The difference between the highest and lowest frequencies available for network signals. 
The term is also used to describe the rated throughput capacity of a given network medium or protocol. 
\end{itemize}
}

\frame{\frametitle{Other Terms}
\begin{itemize} 
\item Bridge \\ 
A network device operating at the first two layers of the OSI model with filtering and forwarding capabilities.
\item Broadcast address \\
An address that allows transmission of a message to all nodes of a network.
\item Congestion \\ Excessive network or internetwork traffic causing a general degradation of service. 
This can be seen in slower response times, longer file transfers and network users becoming less productive due to network delays. 
\end{itemize}
}

\frame{\frametitle{Other Terms}
\begin{itemize} 
\item Carrier Sense Multiple Access with Collision Avoidance (CSMA/CA) \\
An access method in wireless LANs that avoids collision by forcing the stations to send reservation messages when they find the channel is idle.
\item Carrier Sense Multiple Access with Collision Detection (CSMA/CD) \\
An access method in which stations transmit whenever the transmission medium is available and retransmit when collision
occurs.
\item Collision \\
The event that occurs when two transmitters send at the same time on a channel designed for only one transmission at a time; data will be destroyed.
\end{itemize}
}


\frame{\frametitle{Other Terms}
\begin{itemize} 
\item Consultative Committee for International Telegraphy and Telephony (CCITT) \\
An international standards group now known as the ITU-T.
\item Defense Advanced Research Projects Agency (DARPA)\\
 A government organization, which, under the name of ARPA, funded ARPANET and the Internet.
\item Ethernet \\
A local area network using the CSMA/CD access method.
\end{itemize}
}


\frame{\frametitle{Other Terms}
\begin{itemize} 
\item Extranet \\
A private network that uses the TCP/IP protocol suite that allows authorized access from outside users.
\item Flooding \\
 Saturation of a network with a message.
intranet A private network that uses the TCP/IP protocol suite.
\item Intranet \\
A private network that uses the TCP/IP protocol suite.
\end{itemize}
}


\frame{\frametitle{Other Terms}
\begin{itemize} 
\item Institute of Electrical and Electronics Engineers (IEEE) \\
A group consisting of professional engineers that has specialized societies whose committees prepare standards in members' areas of specialty.
\item Logical tunnel \\
The encapsulation of a multicast packet inside a unicast packet to enable multicast routing by non-multicast routers.
\item Physical address \\
The address of a device used at the data link layer (MAC address).
\end{itemize}
}


\frame{\frametitle{Other Terms}
\begin{itemize} 
\item Request for Comment (RFC) \\
A formal Internet document concerning an Internet issue.
\item Reverse Address Resolution Protocol (RARP) \\
A TCP/IP protocol that allows a host to find its Internet address, given its physical address.
\item Switch \\
A device connecting multiple communication lines together.
\item Switched Ethernet \\ 
An Ethernet in which a switch, replacing the hub, can direct a transmission to its destination.
\end{itemize}
}



\begin{frame}
\frametitle{References}
\Large
TEXTBOOK: 
\begin{itemize}
\item  <1-> Data Communications and Networking, Behrouz Forouzan, 4th Edition, McGraw-Hill, 2007
 \end{itemize}
\fig{0.3}{forouzan2007}
\end{frame}

\begin{frame}
\frametitle{References}
\Large
SECONDARY SOURCE: 
\begin{itemize}
\item  <1-> TCP/IP Protocol Suite, Behrouz Forouzan, 4th edition, 2010
 \end{itemize}
\fig{0.3}{forouzantcpip}
\end{frame}

\begin{frame}
\frametitle{References}
\Large
SECONDARY SOURCE: 
\begin{itemize}
\item  <1-> Data and Computer Communications, William Stallings, 2007
 \end{itemize}
\fig{0.3}{stallings2007}
\end{frame}
 

\begin{frame}
\frametitle{References}
\Large
SECONDARY SOURCE: 
\begin{itemize}
\item  <1-> CISCO Networking Essentials, Troy McMillan, 2012
 \end{itemize}
\fig{0.3}{mcmillan2012}
\end{frame}

\begin{frame}
\frametitle{References}
\Large
SECONDARY SOURCE: 
\begin{itemize}
\item  <1-> Network Fundamentals, Cisco Networking Academy, 2007
 \end{itemize}
\fig{0.4}{cisco}
\end{frame}


\begin{frame}
\frametitle{Thank you for your attention!}
\fig{0.6}{mthesis}
\end{frame}

\end{document}